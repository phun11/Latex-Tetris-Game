

\documentclass[12pt]{article}
\usepackage{url}
\usepackage[colorlinks=true, urlcolor=blue]{hyperref}

\usepackage{graphicx}



\usepackage[T5]{fontenc}      % hỗ trợ font tiếng Việt
\usepackage[utf8]{inputenc}   % mã hóa UTF-8
\usepackage[vietnamese]{babel} % gói ngôn ngữ tiếng Việt

\usepackage{array}
\usepackage{geometry}
\geometry{a4paper, margin=1in}
\usepackage{longtable}



\begin{document}

\begin{center}
{\Large \textbf{HỢP ĐỒNG THÀNH LẬP NHÓM}}\\[6pt]
\end{center}

\noindent \textbf{Thời gian thành lập:} 24-11-2025

\noindent \textbf{Tên nhóm:} 5 chú cá

\section*{Thành viên:}

\begin{longtable}{|c|m{5cm}|c|}
\hline
\textbf{STT} & \textbf{Họ và tên} & \textbf{MSSV} \\
\hline
1 & Man Mỹ Phương & 24521414 \\
\hline
2 & Trần Thu Phương & 24521419 \\
\hline
3 & Thành Công Vinh & 24522027 \\
\hline
4 & Nguyễn Văn Vỹ & 24522063 \\
\hline
5 & Nguyễn Trần Duy Anh & 205220393 \\
\hline
\end{longtable}

\section*{Nguyên tắc làm việc nhóm:}

 Điều 1: Ứng xử văn minh, lịch sự và tôn trọng tất cả các thành viên trong nhóm, trên tinh thần bình đẳng và hợp tác.

Điều 2: Tích cực, chủ động trong việc xây dựng và đóng góp ý kiến cho hoạt động của nhóm.

Điều 3: Có tinh thần trách nhiệm cao với các công việc được giao.

Điều 4: Hoàn thành đầy đủ và đúng thời hạn mọi nhiệm vụ đã được phân công theo kế hoạch nhóm.

Điều 5: Giữ tinh thần đoàn kết, sẵn sàng giúp đỡ và hỗ trợ lẫn nhau trong quá trình làm việc.

Điều 6: Luôn đúng giờ trong các buổi học và họp nhóm. Nếu có lý do đi trễ hoặc vắng mặt, cần báo trước cho nhóm.

Điều 7: Khi gặp khó khăn hoặc trở ngại khiến bản thân không thể hoàn thành công việc đúng hạn, thành viên phải báo trước với nhóm ít nhất 3 ngày để cùng tìm hướng giải quyết.

Điều 8: Nếu có bất đồng ý kiến sẽ giải quyết theo nguyên tắc biểu quyết đa số.

Điều 9: không được gây chia rẽ, mất đoàn kết giữa các thành viên trong nhóm.

Điều 10: Phải hoàn thành nhiệm vụ được giao đúng thời hạn mà không có lý do chính đáng.

Điều 11: Đi học hoặc họp nhóm đúng giờ, tham gia các buổi họp nhóm.

Điều 12: Báo trước cho nhóm về việc vắng mặt hoặc không thể hoàn thành công việc đúng thời hạn theo quy định.

\section*{Không gian trao đổi và giao tiếp}

\subsection*{1. Slack}

Slack là một nền tảng giao tiếp và làm việc nhóm chuyên dụng, được sử dụng rộng rãi trong các doanh nghiệp và môi trường giáo dục.

Công cụ này cho phép sinh viên tạo các kênh (channel) riêng biệt cho từng môn học, dự án hoặc chủ đề, giúp việc trao đổi thông tin trở nên có tổ chức và hiệu quả hơn.

Slack hỗ trợ nhắn tin nhanh, gửi tệp, thực hiện cuộc gọi video, gắn thẻ thành viên và lưu trữ lịch sử trò chuyện, tạo điều kiện thuận lợi cho việc theo dõi tiến độ công việc và đảm bảo tính minh bạch trong giao tiếp nội bộ.

Một ưu điểm nổi bật của Slack là khả năng tích hợp linh hoạt với nhiều công cụ khác như Google Drive, Trello, Zoom, giúp việc quản lý và phối hợp trong các dự án nhóm phức tạp trở nên dễ dàng và chuyên nghiệp hơn.

Link: \url{https://app.slack.com/client/T09DLM2RRNX/C0A00H16K9A}

\subsection*{2. Zalo}

Zalo là nền tảng nhắn tin và trao đổi thông tin phổ biến tại Việt Nam, hỗ trợ nhóm làm việc giao tiếp nhanh chóng và thuận tiện. Với khả năng gửi tin nhắn tức thời, chia sẻ tài liệu, hình ảnh, tạo nhóm trò chuyện và thực hiện cuộc gọi, Zalo giúp các thành viên cập nhật tiến độ, phân công nhiệm vụ và trao đổi ý tưởng hiệu quả. Nhờ tính đơn giản, dễ truy cập trên cả điện thoại và máy tính, Zalo trở thành công cụ giao tiếp linh hoạt hỗ trợ quá trình phối hợp khi thực hiện báo cáo nhóm.

Link: \url{https://zalo.me/g/glejxa926}

\subsection*{3. Google meet}

Link: \url{https://meet.google.com/hdn-wtzq-fmo}

\subsection*{4. Google Docs (Tài liệu Google)}

Google Docs là một công cụ cộng tác trực tuyến phổ biến, được sử dụng rộng rãi trong môi trường học tập và làm việc nhóm.

Nền tảng này cho phép nhiều người dùng cùng chỉnh sửa tài liệu trong cùng một thời điểm, theo dõi lịch sử thay đổi và đưa ra nhận xét trực tiếp trên nội dung đang làm việc.

Tất cả các chỉnh sửa đều được tự động lưu trữ, giúp hạn chế rủi ro mất dữ liệu.

Ngoài ra, Google Docs còn tích hợp chặt chẽ với các công cụ khác trong hệ sinh thái Google như Google Sheets, Google Slides và Google Drive, cho phép nhóm soạn thảo báo cáo, phân tích số liệu, thiết kế bài thuyết trình và lưu trữ tài liệu chung một cách hiệu quả và thuận tiện.

Link: \href{https://docs.google.com/document/d/1do6Aa7rRsqR6zIB9B6g-Lp9p4E3hvgJRdrln6y5CTVo}{Dàn ý Tetris Game 1}



\subsection*{5. Word}

Công cụ soạn thảo văn bản giúp mỗi cá soạn thảo nội dung được phân công trong phần làm việc nhóm. Dữ liệu của cá nhân được lưu lại ngay trên máy để dễ dàng chỉnh sửa khi mà không cần mạng internet.

\subsection*{6. Overleaf}

Overleaf là một nền tảng soạn thảo LaTeX trực tuyến cho phép nhiều thành viên trong nhóm làm việc đồng thời trên cùng một tài liệu. Môi trường biên dịch tích hợp sẵn giúp hiển thị kết quả theo thời gian thực mà không cần cài đặt thêm phần mềm. Nhờ hỗ trợ cộng tác trực tiếp, quản lý phiên bản tự động và khả năng chia sẻ dễ dàng qua liên kết hoặc email, Overleaf rất phù hợp cho các nhóm thực hiện báo cáo, bài nghiên cứu hay luận văn yêu cầu định dạng LaTeX thống nhất và chính xác.

Link: \url{https://github.com/phun11/Latex-Tetris-Game}

\noindent \url{https://www.overleaf.com/read/brsdkxmtznzz#54f937}

\subsection*{7. Github}

GitHub là nền tảng quản lý mã nguồn sử dụng hệ thống kiểm soát phiên bản Git, cho phép nhóm lưu trữ, theo dõi và đồng bộ hóa toàn bộ tệp dự án trong một kho lưu trữ chung. Nhờ cơ chế commit, branch và pull request, các thành viên có thể làm việc song song mà vẫn đảm bảo kiểm soát được lịch sử thay đổi và tránh xung đột nội dung. GitHub đặc biệt hữu ích khi kết hợp với Overleaf để quản lý nguồn LaTeX, giúp nhóm duy trì cấu trúc tài liệu rõ ràng, minh bạch và dễ dàng phối hợp trong quá trình viết báo cáo.

Link: \url{https://github.com/phun11/Tetris-Game}

\section*{Mục đích thành lập nhóm}

Nâng cao kỹ năng làm việc nhóm và các kỹ năng mềm khác cho các thành viên.

Hỗ trợ, động viên lẫn nhau để hoàn thành tốt nhiệm vụ và đạt kết quả cao trong môn học.

\section*{VII. Đánh giá công tác làm việc nhóm}

\subsection*{1. Bảng tiêu chí đánh giá}

\begin{longtable}{|m{4cm}|m{3cm}|m{2.5cm}|m{3cm}|m{2.5cm}|}
\hline
\textbf{Tiêu chí đánh giá} & \textbf{Xuất sắc} & \textbf{Tốt} & \textbf{Bình thường} & \textbf{Kém} \\
\hline
1. Thái độ làm việc & Chủ động nhận và hoàn thành tốt nhiệm vụ được giao & Hoàn thành nhiệm vụ được giao đúng yêu cầu & Hoàn thành nhiệm vụ sau khi được nhắc nhở & Không hoàn thành nhiệm vụ được giao \\
\hline
2. Quản lý thời gian & Hoàn thành nhiệm vụ trước hạn hoặc đúng hạn theo lịch phân công & Hoàn thành nhiệm vụ đúng hạn hoặc trễ không quá 12 giờ theo lịch phân công & Hoàn thành nhiệm vụ trễ nhưng không quá 24 giờ theo lịch phân công & Không hoàn thành nhiệm vụ hoặc trễ quá 24 giờ theo lịch phân công \\
\hline
3. Nghiên cứu tài liệu & Chủ động tìm kiếm, tổng hợp và chia sẻ tài liệu hữu ích cho nhóm & Tìm kiếm tài liệu phục vụ phần việc được giao & Có nghiên cứu nhưng chưa đầy đủ, cần hỗ trợ thêm & Không tìm hiểu hoặc không đóng góp tài liệu cho nhóm \\
\hline
4. Phối hợp làm việc nhóm & Chủ động hỗ trợ, hợp tác tốt với các thành viên khác & Phối hợp tốt trong phạm vi công việc được giao & Chỉ hợp tác khi được yêu cầu & Thiếu tinh thần hợp tác, làm việc tách biệt \\
\hline
5. Giải quyết vấn đề được giao & Chủ động tìm kiếm giải pháp, đề xuất hướng xử lý hiệu quả & Tìm kiếm giải pháp phù hợp theo phân công & Tham gia góp ý nhưng chưa tìm ra hướng giải quyết cụ thể & Không tham gia giải quyết vấn đề \\
\hline
6. Kết nối, giao tiếp hiệu quả & Chủ động liên lạc, trao đổi thường xuyên, rõ ràng với nhóm & Luôn giữ liên lạc và phản hồi kịp thời & Thỉnh thoảng phản hồi chậm hoặc mất liên lạc trong 24 giờ & Không liên lạc, không phản hồi \\
\hline
7. Đóng góp ý kiến & Chủ động đưa ra ý kiến, sáng kiến giúp cải thiện chất lượng công việc & Đưa ra ý kiến khi liên quan đến phần việc cá nhân & Chỉ phát biểu khi được yêu cầu & Không tham gia đóng góp ý kiến \\
\hline
8. Hoàn thành công việc đúng hạn & Hoàn thành công việc vượt mong đợi, chất lượng cao & Hoàn thành công việc đúng hạn, đạt yêu cầu & Hoàn thành công việc trễ nhưng đảm bảo chất lượng & Không hoàn thành công việc đúng thời hạn hoặc không đạt yêu cầu \\
\hline
\end{longtable}

\subsection*{2. Cam kết nhóm}

Tất cả thành viên trong nhóm cam kết hợp tác với nhau trên tinh thần vui vẻ, tôn trọng và hết mình hỗ trợ lẫn nhau.

Các thành viên không vi phạm điều lệ nhóm, hoàn thành tốt nhiệm vụ cá nhân và đóng góp tích cực cho mục tiêu chung mà nhóm đã đề ra.

Thành viên nhóm đã đọc, hiểu và đồng ý với các điều khoản, hướng dẫn được nêu trong hợp đồng này.

Hợp đồng lập nhóm đã được thông qua và ký kết.

\vspace{1cm}

\noindent Thành phố Hồ Chí Minh, ngày 28 tháng 11 năm 2025

\vspace{1.5cm}


% Dòng chữ nằm bên phải + tăng 2 size
\begin{flushright}
    {\Large \textit{Các thành viên ký tên:}}
\end{flushright}

\vspace{0.7cm}

% Chữ ký chia thành 2 hàng, mỗi hàng 2 người
\begin{center}
\begin{tabular}{ccc}
    \includegraphics[width=4cm]{mphun.jpg} &
    \includegraphics[width=4cm]{duyanh.jpg} \\[1.2cm]
    \includegraphics[width=4cm]{congvinh.jpg} & 
    \includegraphics[width=4cm]{thuphun.jpg} &
    \includegraphics[width=4cm]{vy.jpg} \\
\end{tabular}
\end{center}



\vspace{3cm}


\section*{B. GIỚI THIỆU VÀ HƯỚNG DẪN CHƠI GAME}

\subsection*{Lời mở đầu và giới thiệu chung}

\subsubsection*{1. Lời chào và bối cảnh}

Chào mừng bạn đến với phiên bản Tetris của chúng tôi!

Tetris là trò chơi xếp khối kinh điển, được yêu thích trên toàn thế giới. Đặc biệt, trò chơi này phù hợp với mọi lứa tuổi bởi nó không những mang lại sự thư giãn, giải trí sau những giờ học hoặc làm việc căng thẳng mà còn thách thức tư duy, cải thiện sự tập trung, trí nhớ và rèn luyện phản xạ, tư duy chiến lược.

Tetris là một trò chơi điện tử giải đố kinh điển, phổ biến nhất mọi thời đại được phát triển bởi Alexey Pajitnov, một kỹ sư phần mềm người Liên Xô. Tên "Tetris" là sự kết hợp của "tetra" (nghĩa là số bốn trong tiếng Hy Lạp) và "tennis" (môn thể thao yêu thích của tác giả).

Trong lối chơi Tetris điển hình, các hình khối tetromino đầy màu sắc rơi từ trên xuống, nhiệm vụ của bạn là phải sắp xếp chúng thành một chồng. Khi một hàng hoặc cột của bảng trò chơi được lấp đầy, nó sẽ biến mất và điểm số của bạn sẽ tăng lên.

Phiên bản của chúng tôi sở hữu lối chơi gần giống bản cổ điển nhưng mang đến trải nghiệm mượt mà, hiện đại, đồng thời còn bổ sung một số tính năng mới như Next Queue và lưu điểm cao nhất.

\section*{II. Cài đặt và thiết lập}

\subsection*{1. Cài đặt}

Quy trình cài đặt gồm 3 bước đơn giản:

\begin{itemize}
    \item \textbf{Bước 1: Truy cập và Tải về.} Truy cập vào đường dẫn GitHub của dự án. Tại giao diện chính, tìm và chọn tải xuống tệp nén của trò chơi.
    \item \textbf{Bước 2: Giải nén.} Sau khi tải xong, nhấp chuột phải vào tệp tin vừa tải về và giải nén toàn bộ thư mục trò chơi.
    \item \textbf{Bước 3: Mở game.} Truy cập vào thư mục vừa giải nén, tìm file thực thi của trò chơi và nhấp đúp chuột để bắt đầu chơi.
\end{itemize}

\subsection*{2. Tối ưu}

Sau khi mở game, để có trải nghiệm chơi mượt mà và tập trung nhất, bạn nên kiểm tra nhanh các thiết lập:

\begin{itemize}
    \item \textbf{Đồ họa \& Hiển thị:} Trò chơi mặc định chạy ở chế độ cửa sổ tiêu chuẩn. Đảm bảo độ phân giải màn hình của bạn phù hợp để nhìn rõ các khối gạch và điểm số.
    \item \textbf{Âm thanh:} Kiểm tra loa hoặc tai nghe để tận hưởng nhạc nền và hiệu ứng âm thanh (SFX) khi xếp gạch. Bạn có thể điều chỉnh âm lượng hệ thống sao cho vừa đủ nghe, giúp tăng sự tập trung.
\end{itemize}

\section*{V. Hướng dẫn chơi theo tiến trình}

\subsection*{1. Bắt đầu}

Khi mở game, bạn sẽ thấy giao diện tuỳ chọn các chế độ phổ biến bao gồm:

\begin{itemize}
    \item \textbf{Marathon}: chơi đến khi thua, mục tiêu đạt mức điểm nhất định.
    \item \textbf{Sprint / 40 Lines}: hoàn thành 40 hàng nhanh nhất.
    \item \textbf{Ultra}: ghi điểm trong 2 phút.
    \item \textbf{Endless}: chơi vô hạn đến khi thua.
\end{itemize}

Sau khi chọn chế độ, bạn sẽ cần chọn độ khó (level càng cao, tốc độ rơi gạch càng nhanh). Có 3 mức độ:

\begin{itemize}
    \item Easy
    \item Medium
    \item Hard
\end{itemize}

\textbf{Khuyến nghị cho người mới:} nên bắt đầu với chế độ \textbf{Marathon} hoặc \textbf{Endless} ở mức \textbf{Easy} để làm quen tốc độ và điều khiển.

\subsection*{2. Chơi game}

Mục tiêu chính của trò chơi là xếp các khối để tạo thành một hoặc nhiều hàng hoàn chỉnh. Khi lấp đầy hàng, chúng sẽ biến mất và người chơi nhận điểm dựa trên số hàng xoá liên tiếp và tốc độ chơi.

\medskip

Theo dõi khu vực \textbf{Next Queue} (hiển thị khối sẽ xuất hiện tiếp theo, nằm bên phải màn hình) để lập kế hoạch sắp xếp. Điều này giúp chuẩn bị vị trí cho các khối dài để tạo \textbf{Tetris} (xoá 4 hàng cùng lúc).

\medskip

\noindent \textbf{Các điều khiển quan trọng:}

\begin{itemize}
    \item \textbf{Move Left / Right}: di chuyển khối sang trái / phải.
    \item \textbf{Rotate}: xoay khối theo chiều kim đồng hồ.
    \item \textbf{Soft Drop}: tăng tốc độ rơi nhưng vẫn điều khiển được.
    \item \textbf{Hard Drop}: thả khối xuống đáy ngay lập tức, ghi thêm điểm.
\end{itemize}

\textit{Lưu ý:} Người mới nên dùng Soft Drop để dễ kiểm soát; khi quen hãy dùng Hard Drop để tăng tốc độ chơi và ghi điểm.

\medskip

\noindent \textbf{Cách ghi điểm:}

\begin{itemize}
    \item Clear 1 line → điểm thấp.
    \item Clear 2 lines → điểm cao hơn.
    \item Clear 3 lines → nhiều điểm hơn nữa.
    \item Clear 4 lines (\textbf{Tetris}) → điểm rất cao.
    \item Combo liên tục → \textit{bonus points}.
    \item Hard drop / Soft drop → nhận thêm điểm cho mỗi khối.
\end{itemize}

\textbf{Lưu ý quan trọng:} Trò chơi sẽ \textbf{tăng level} sau mỗi số hàng nhất định (thường 10). Level càng cao, gạch rơi càng nhanh. Người mới nên giữ stack thấp để tránh áp lực khi tốc độ tăng.

\subsection*{3. Kết thúc trò chơi}

Trò chơi kết thúc khi:

\begin{itemize}
    \item gạch chạm đỉnh,
    \item hết thời gian,
    \item hoàn thành số hàng quy định (tuỳ chế độ).
\end{itemize}

Màn hình kết thúc hiển thị:

\begin{itemize}
    \item Tổng điểm
    \item Thời gian (tính đến mili giây)
    \item Cấp độ đã đạt được
    \item Tổng số hàng đã xoá
    \item Combo cao nhất
\end{itemize}

Các lựa chọn sau khi kết thúc:

\begin{itemize}
    \item Restart (chơi lại)
    \item Lưu và cập nhật điểm (kỷ lục)
    \item Xem bảng xếp hạng
\end{itemize}


\subsection*{VII. Lời kết và lời mời}

Tetris không chỉ là trò chơi giải trí, mà còn là bài tập rèn luyện trí não, giúp bạn thư giãn sau những giờ học tập hay làm việc căng thẳng.

Hãy tải game và bước vào hành trình xếp khối đầy thử thách! Chinh phục từng khối, phá vỡ kỷ lục của chính bạn và trở thành bậc thầy Tetris.

\end{document}